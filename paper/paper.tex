% !TEX TS-program = xelatex
%
% Created by Chris on 2020-07-19.
% Copyright (c) Chris von Csefalvay, 2020.
\documentclass{article}
\usepackage{amsmath}
\usepackage{polyglossia}
\usepackage{hyperref}

% Bibliography styling
\usepackage[super,square,sort&compress,numbers]{natbib}
\bibliographystyle{unsrtnat}
\usepackage{url}
\urlstyle{same}

% Language and hyphenation
% \usepackage[english]{babel}
\usepackage[htt]{hyphenat}

% Graphics
\usepackage{graphicx}

\hypersetup
{
  pdftitle   = {Marginal dynamics of social distancing in COVID-19},
  pdfauthor  = {Chris von Csefalvay}
}

\title{Marginal dynamics of social distancing in COVID-19}
\author{Chris von Csefalvay}

\begin{document}

\maketitle

\begin{abstract}
    Abstract
\end{abstract}

\section{Introduction} % (fold)
\label{sec:introduction}
Where an infectious disease is not amenable to population-level prevention through vaccination and risks are non-trivial, non-pharmaceutical interventions (NPIs) remain the principal tool of public health to respond to an outbreak. This is the case with novel infectious diseases that have no specific treatment and no prophylactic (vaccine) available. In the absence of pharmaceutical interventions of proven effectiveness, in particular prophylactically, the main public health response to the emerging pandemic of COVID-19, a viral syndrome caused by the (+)ssRNA virus SARS-CoV-2 (order \emph{Nidovirales}, family \emph{Coronaviridae}, genus \emph{Betacoronavirus}, subgenus \emph{Sarbecovirus}), has rested principally on NPIs, first and foremost social distancing and ancillary steps intended to facilitate that.

In general, any course of conduct that reduces the encounter rate between an individual and other individuals can be considered a form of social distancing. This may be brought about through limiting public facilities for such encounters ('lockdowns'), through limiting individual gatherings by size ('large-gathering bans') and through encouraging individual social distancing. From the perspective of game theory, social distancing can be viewed as a non-cooperative game of a population $P_{1 \ldot n}$ of size $n$, where at any given time $t \in [t_0, t_{e}]$, the strategy adopted by $p_i$ is denoted as $\sigma (p_i, t)$. For the sake of simplicity, we assume that distancing is either not exercised at all or perfectly exercised, i.e. $\sigma (p_i, t) \in \{ 0, 1 \}$. Then, for the entire population, the overall strategy can be described as 

\begin{equation}
	\bar{\sigma}(P, t) = \frac{\displaystyle \sum_{i = 0}^n \sigma(p_i, t)}{n}
	\label{eq:overall_strategy}
\end{equation}

\noindent and for the entire time period, from $t_0$ to the endpoint $t_e$ (which may be eradication, elimination, natural extinction of the pathogen, the availability of a vaccine or a combination thereof), for discrete time $t$,

\begin{equation}
	\bar{\bar{\sigma}}(P) = \frac{\displaystyle \sum_{i = 0}^n \displaystyle \sum_{j = 0}^{t_e - t_0} \sigma(p_i, j)}{n (t_e - t_0)}
	\label{eq:overall_strategy_over_time}
\end{equation}

However, with each course of conduct, there is associated a cost $J(\sigma)$. For simplicity's sake, let us consider these costs to be governed by the following three precepts given a pathogen with fixed characteristics ($R_0$, transmission potential, risks \&c.). Let $\bar{\delta}(P)$ equal the proportion of persons $p \in P$ adopting strategy $\sigma_{\delta}$. It then holds that: 

\begin{enumerate}
	\item A person $p_i \in P$ opting for strategy $\delta$ (social distancing) will incur $c_d$, the immediate costs of distancing. These may be social (lessened social interaction), psychological (lessened access to support systems), economic (lower access to facilities to earn) or simple matters of convenience (access to amenities). While $c_d$ is somewhat dependent on $\bar{\bar{(P)}}$ (thus not distancing does not yield a benefit to a lone social distancer in terms of access to amenities if all of the latter are closed), it can be assumed to be largely constant.
	\item Compared to a person opting for strategy $\delta$, a person $p_i \in P$ opting for strategy $\lnot \delta$ will incur a relative additional risk $f_r(\bar{\delta}(P), t)$, i.e. contingent on the population's behaviour at time $t$.
	\item Finally, a person $p_i \in P$ will regardless of their individual choice (at least at non-trivial population sizes) gain a protective benefit, which is a function of the population's adherence to social distancing. We will denote this term as $f_c(\bar{\delta}(P), t)$.
\end{enumerate}

Then, assigning the variable $c_s$ to represent the cost of illness (economic loss, medical costs, long-term health risks), we can express the cost functions of each strategy, $\sigma_{\delta}$ (social distancing) and $\sigma_{\lnot \delta}$ (no social distancing), as

\begin{equation}
	J_{\sigma_{\delta}}(p_i, t) = c_d - f_c(\bar{\delta}(P), t)
\end{equation}

\noindent and

\begin{equation}
	J_{\sigma_{\lnot \delta}}(p_i, t) = f_r(\bar{\delta}(P), t) c_s - f_c(\bar{\delta}(P), t)
	\label{eq:risk}
\end{equation}

\noindent And consequently, it holds for the population level cost $\bar{J}$ at time $t$ that

\begin{equation}
	\bar{J}(P, t) = \frac{\displaystyle \sum_{i=0}^{\delta(P) n} J_{\sigma_{\delta}}(p_i, t) + \displaystyle \sum_{j=0}^{1-\delta(P) n} J_{\sigma_{\lnot \delta}}(p_j, t)}{n}
	\label{eq:cost_eqn}
\end{equation}

\noindent and for a population over time $t_0$ to $t_e$, 

\begin{equation}
	\bar{\bar{J}}(P) = \int_{t_0}^{t_e} \bar{J}(P, t) dt
\end{equation}

As Reluga (2010) notes, the effect of social distancing diminishes with the increase in participants, the phenomenon of diminishing returns.\cite{reluga2010game} Thus, as the number of individuals in $P$ opting for social distancing increases, the marginal increase by another social distancer diminishes. This can be represented by a discount factor $h$, to yield

\begin{equation}
	\bar{J}(P) = \int_{t_0}^{t_e} e^{-ht} \frac{\delta(P) J_{\sigma_{\delta}}(P, t) + (1 - \delta(P)) J_{\sigma_{\lnot \delta}}(P, t)}{n}
\end{equation}

However, because individual decisions affect the overall gain (due to the component dependent on $\delta(P, t)$), our principal concern is not with individual action but with analysis of such strategies on a population level. This paper will in the following conceptualise infectious disease in a population as a differential game over a differential equation form of the compartmental model first described by Kermack and McKendrick\cite{kermack1927contribution} and, since its publication in 1927, widely adapted and adopted.\cite{vstvepan2007kermack,roberts1999kermack,capasso1978generalization}

% section introduction (end)

\section{Methods} % (fold)
\label{sec:methods}

\subsection{The ordinary differential equations of disease dynamics} % (fold)
\label{sub:the_ordinary_differential_equations_of_disease_dynamics}

Given a population of $n$ under the assumption that reinfection is impossible (as, e.g., in the case of measles) or rare (as is the case for SARS-CoV-2\cite{edridge2020human,deng2020primary,bao2020reinfection}), and neglecting for the time being the vital dynamics (birth, unrelated death, migration) of the population, the dynamics of any population can be modelled as a system of ordinary differential equations

\begin{equation}
	\begin{aligned}
		\frac{dS}{dt} = - \frac{\beta S I}{n} 								\\
		\frac{dI}{dt} = \frac{\beta S I}{n} - \gamma I 						\\
		\frac{dR}{dt} = \gamma I
	\end{aligned}
	\label{eq:sir_equation}
\end{equation}


\begin{figure}
	\includegraphics[width=\linewidth]{figures/fig1-odes}
	\caption{Some quantitative solutions for the SIR model's population dynamics over values of $R_0$ between 1.5 and 6.5, and values of $\gamma$ between $1/4.6$ and $1/12.4$ over a base population of 10,000 and a seed population of 1\% infected initially. For each plot, $\beta$ is inferred from an $R_0$ value of 2.67, based on Liu et al. (2020),\cite{liu2020reproductive} and the $\gamma$ parameter. The susceptible population is displayed in blue, while infected/infectious cases are marked in red and recovered cases in green.}
	\label{fig:ode_solutions}
\end{figure}

\noindent under the assumption of $S + I + R = 0$, where $S$ represents susceptible individuals, $I$ represents infected/infectious individuals and $R$ accounts for removed individuals (mortality and recovery to immunity). The fraction $\frac{\beta}{\gamma}$ equals the basic reproduction number, $R_0$. For SARS-CoV-2, estimates of $R_0$ range from 1.4 to 6.49, with studies that relied on statistical estimation of $R_0$ ranging from 2.20 to 3.58, with an average of 2.67\cite{liu2020reproductive} $\gamma$, on the other hand, can be estimated as the invers of the number of days of illness, which can be approximated as $8.5 \pm 3.9$ days.\cite{pan2020clinical,liu2020risk} Figure~\ref{fig:ode_solutions} describes some analytical solutions for the differential equations of Equation~\eqref{eq:sir_equation} over a range of plausible values of $R_0$ and $\gamma$ under the assumption of an $R_0$ of 2.67. Even in the absence of firm evidence as to whether SARS-CoV-2 infection followed by recovery would engender lifelong immunity or not,\cite{roy2020covid,ota2020will,lin2020duration} it can be assumed in the short term -- based on evidence from MERS-CoV and SARS-CoV -- that in the short term, survivors remain immune,\cite{prompetchara2020immune} and consequently the number of individuals in $R$ does not decrease. Numerical solutions to this system of differential equations have been calculated using \texttt{odepack} via \texttt{SciPy 1.5.1}\cite{virtanen2020scipy} on Python 3.6, and are presented as Figure~\ref{fig:ode_solutions} for a range of values of $R_0$ and $\gamma$.

% subsection the_ordinary_differential_equations_of_disease_dynamics (end)

\subsection{Population strategy contingent solutions to population dynamics} % (fold)
\label{sub:population_strategy_contingent_solutions_to_population_dynamics}

\begin{figure}
	\includegraphics[width=\linewidth]{figures/fig3-SIR-by-delta}
	\caption{Contour diagrams of numeric solutions for susceptible, infectious and removed ($S$, $I$ and $R$) compartment sizes for a base population of 10,000 individuals with a seed population of 1\% infected, under the assumption of an $R_0$ of 2.67 and $\gamma$ of $\frac{1}{8.5}$. As this figure indicates, the 'critical mass' of social distancing takes place in the $\delta$ range of 0 to 0.4, and thus even modest increases in social distancing participation at low levels can make a significant difference in the number of infectious cases.}
	\label{fig:fig3-SIR-by-delta}
\end{figure}

Given a population that then adopts a sum strategy $\bar{\sigma}(P, t)$ at time $t$ that results in $\delta(P, t)$ adherence (or in Reluga's terms, investment) to social distancing, the flow from $S$ to $I$ is reduced by a corresponding factor. This allows us to rewrite Equation~\eqref{eq:sir_equation} so that for a society level strategy $\bar{\sigma}$ yielding $\delta$, the populations can be characterised as

\begin{equation}
	\begin{aligned}
		\frac{dS}{dt} = - \frac{\beta S I - \delta \beta S I}{n} 			\\
		\frac{dI}{dt} = \frac{\beta S I - \delta \beta S I}{n} - \gamma I	\\
		\frac{dR}{dt} = \gamma I
	\end{aligned}
	\label{eq:sir_with_social_distancing}
\end{equation}

Solutions to this system of differential equations have been calculated and are presented in Figure~\ref{fig:fig3-SIR-by-delta}. Importantly, this allows us to identify the marginal utility $\hat{U}(P, t)$ as

\begin{equation}
	\begin{aligned}
		\hat{U}(P, t) = \frac{\partial I(P, t)}{\partial \delta(t)}
	\end{aligned}
	\label{eq:marginal_utility}
\end{equation}

\noindent i.e. the partial derivative of $I(P, t)$ over $\delta(P, t)$. Thus, for a population-level strategy $\bar{\sigma}(P, t)$ associated with a compliance rate (i.e. the fraction of persons $p_i \in P$ engaged in social distancing at time $t$) of $\delta(P, t)$, there exists a function $\hat{U}(P, t)$ given $\gamma$ and $R_0$ that indicates the marginal utility at any given value of $\delta(P, t)$. This, too, can be numerically ascertained, and is shown on Figure~\ref{fig:marginal_utility}.

\begin{figure}
	\includegraphics[width=\linewidth]{figures/marginal_utility}
	\caption{Marginal utility of social distancing in a population $P$ of $\delta(P)$ adherence over time. The marginal utility is defined as the vertical component of the gradient of infected individuals. The plot draws on a base population of 10,000 individuals with a seed population of 1\% infected, under the assumption of an $R_0$ of 2.67 and $\gamma$ of $\frac{1}{8.5}$.}
	\label{fig:marginal_utility}
\end{figure}

% subsection population_strategy_contingent_solutions_to_population_dynamics (end)

\subsection{Cost, risk and strategy} % (fold)
\label{sub:cost_risk_and_strategy}

Any strategy $\sigma$ has a cost $J(\sigma)$, as stated in Section~\ref{sec:introduction}, and the overall societal cost of $n$ individuals $p_{1 \ldots n} \in P$ each adopting, respectively, strategy $\sigma_{1 \ldots n}$, is $\sum_{i=0}^n J(\sigma_i)$. But since strategies are limited (one may, at any given time, either engage in social distancing or not, assuming for simplicity's sake that those who do so are entirely successful), for any society-level strategy $\bar{\sigma} (t)$ resulting in a level of distancing described by $\delta (t)$, 

\begin{equation}
	\begin{aligned}
		\bar{J}(P, t) = \sum_{i=0}^{\delta(P, t) n} J_{\delta} + \sum_{j=0}^{(1-\delta(P, t)) n} J_{\lnot \delta}
	\end{aligned}
	\label{eq:j_bar}
\end{equation}

\noindent where $J_{\delta}$ is the cost of social distancing for discrete unit time and $J_{\lnot \delta}$ is the cost of not distancing for the same unit time. The latter of these is not constant, as Equation~\eqref{eq:risk} shows, but a function of a constant cost of infection, $c_i$, and the risk of infection ($r_i$), which is contingent on $I(t)$ and $\delta(t)$. Thus, Equation~\eqref{eq:j_bar} can be reformulated (once again, in discrete time) as

\begin{equation}
	\begin{aligned}
		\bar{J}(P, t) = \delta(t) n c_d + (1 - \delta(t)) n J_{\lnot \delta}
	\end{aligned}
\end{equation}

\noindent which expands to

\begin{equation}
	\begin{aligned}
		\bar{J}(P, t) = \delta(t) n c_d + (1 - \delta(t)) n r_i(t) c_i
	\end{aligned}
\end{equation}

For a susceptible individual $p_i \in S$, the risk of infection $r_i(t)$ in discrete time is the proportional likelihood of infection, or in other words, 

\begin{equation}
	\begin{aligned}
		r_i(t) = \frac{\beta S(t) I(t)}{n^2}
	\end{aligned}
\end{equation}

While quantification of costs of illness is difficult, quantification of the economic, social and emotional costs of social distancing is even harder to grasp. However, we may, for a range of cost fractions $\frac{c_d}{c_i}$, estimate values of $\delta(t)$ that denote optimal strategic spaces. Thus, 

\begin{equation}
	\begin{aligned}
		\frac{c_d}{c_i}(P, t) = \frac{(1 - \delta(t)) R_0 \gamma S(t) I(t)}{\delta(t) n^2}
	\end{aligned}
\end{equation}

\begin{figure}
	\includegraphics[width=\linewidth]{figures/cost_fraction}
	\caption{Cost fraction $\frac{c_d}{c_i}$ of social distancing in a population $P$ of $\delta(P)$ adherence over time, based on a population of 10,000 individuals with a seed population of 1\% infected, under the assumption of an $R_0$ of 2.67 and $\gamma$ of $\frac{1}{8.5}$. The contour lines indicate the cost fraction, i.e. what fraction of the cost of social distancing $c_d$ the cost of illness $c_i$ must be in order to make not distancing a preferred strategy.}
	\label{fig:cost_fraction}
\end{figure}

As numerical estimation of this cost fraction (Figure~\ref{fig:cost_fraction}) shows that for most cases, not distancing can only be a feasible strategy if the cost of distancing very significantly outweighs the cost of illness, often by as much as four orders of magnitude. Since this is notwithstanding risk (i.e. the comparison needs to be made between the absolute cost of infection and the absolute cost of distancing, rather than the probability/risk-adjusted figures).

% subsection cost_risk_and_strategy (end)

% section methods (end)

\section{Results} % (fold)
\label{sec:results}




This risk, viewed as a function of $R_0$ and $\delta(P)$ in Figure~\ref{fig:risk_and_compliance}, indicates the somewhat counterintuitive finding that social distancing becomes more, 


% section results (end)

\bibliography{bibliography}

\end{document}
